% !TEX TS-program = xelatex
% !TEX encoding = UTF-8 Unicode
% !Mode:: "TeX:UTF-8"

\documentclass{resume}
\usepackage{zh_CN-Adobefonts_external} % Simplified Chinese Support using external fonts (./fonts/zh_CN-Adobe/)
% \usepackage{NotoSansSC_external}
% \usepackage{NotoSerifCJKsc_external}
% \usepackage{zh_CN-Adobefonts_internal} % Simplified Chinese Support using system fonts
\usepackage{linespacing_fix} % disable extra space before next section
\usepackage{cite}

\begin{document}
\pagenumbering{gobble} % suppress displaying page number

\name{魏强}

\basicInfo{
  \email{772122679@qq.com} \textperiodcentered\ 
  \phone{(+86) 155-296-20523} \textperiodcentered\ 
  \linkedin[陕西西安]{https://www.linkedin.com/in/billryan8}}
 
\section{\faGraduationCap\  教育背景}
\datedsubsection{\textbf{北京邮电大学}, 北京}{2013 -- 2017}
\textit{学士}\ 电子科学与技术
\textit{主修课程}\ java 程序语言设计、C++、电子电路分析等

\section{\faUsers\ 工作项目经历}
\datedsubsection{\textbf{中移在线服务有限公司} 北京/郑州}{2017年8月 -- 2019年1月}
NLP 算法工程师
\begin{实现垃圾短信和邮件分类算法}
  \item 基于tfidf/贝叶斯传统机器学习分类算法模型
  \item 实现随机删除拼接等数据增强方案
  \item 最终准确率 97%
\end{itemize}

\begin{智能工单分类和来电原因分析算法调研和实现以及模型封装上线}
  \item 工单分类及来电原因分析算法主要采用text CNN以及rncc深度学习算法实现分类
  \item 数据的读取及预测均为多线程,上线主要使用docker,redis,zookeeper,kafka部署
  \item 最终准确率 90%
\end{itemize}

\begin{完成短信内容实体识别和信息抽取算法工作}
  \item 尝试BIEO 标注方案和深度学习方法lstm+crf算法模型和指针抽取两种算法方案
  \item F1 0.90+
\end{itemize}

\begin{辅助导航对话系统上线}
\end{itemize}

% Reference Test
%\datedsubsection{\textbf{Paper Title\cite{zaharia2012resilient}}}{May. 2015}
%An xxx optimized for xxx\cite{verma2015large}
%\begin{itemize}
%  \item main contribution
%\end{itemize}

%% Reference
%\newpage
%\bibliographystyle{IEEETran}
%\bibliography{mycite}
\end{document}
